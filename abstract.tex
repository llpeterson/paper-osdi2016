\begin{abstract}
   This paper presents a new type of
   wide-area service called a \textit{storage kernel}.  Unlike
   wide-area storage systems, a storage kernel offers a user-programmable
   data-plane, where each user controls how reads and writes affect their
   data independently of applications.  By doing so, users are able to leverage existing
   storage services for availability and persistence, and implement arbitrarily
   complex storage functionality on top of them in a portable manner.
   This isolates applications from underlying
   service behavior, makes services functionally interchangeable, and
   separate applications logic from both storage and user authentication logic.

   The purpose of a storage kernel is to facilitate the construction of
   wide-area applications that keep their state on untrusted user storage, while remaining
   agnostic to their implementations.  This not only puts users back in control
   of their data, but also reduces application complexity--in some cases,
   obviating the need for application-specific servers entirely.
   
   To validate the concept, we present a prototype storage kernel called
   Syndicate, which uses a sovereign identity system to
   implement a runtime for these applications without introducing a single point of trust.
   We show that Syndicate enables the construction of three non-trivial wide-area
   applications using freely-available cloud storage services.  Syndicate does
   so with acceptable performance overhead, while only requiring a small amount
   amount of reusable storage logic written in Python.
   
\end{abstract}

